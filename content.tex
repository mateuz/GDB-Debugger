\section{\scshape GDB}
\frame{\tableofcontents[
    currentsection,
    currentsubsection,
    subsectionstyle=show/shaded/hide
]}

\subsection{ O que é gdb?}
\frame{\tableofcontents[
    currentsection,
    currentsubsection,
    subsectionstyle=show/shaded/hide
]}
\begin{frame}{O que é gdb?}
    \begin{itemize}
        \item É um depurador :o 
        \item Um depurador permite a visualização do que esta acontecendo dentro de um programa enquanto ele esta sendo executado.
        \item Permite quatro tipos de operações principais:
        \begin{itemize}
            \item Iniciar o programa especificando qualquer coisa que possa afetar seu comportamento;
            \item Fazer o programa parar em condições especificas;
            \item Examinar o que aconteceu quando o programa foi interrompido;
            \item Modificar informações no programa, de modo que seja possível experimentar os efeitos de uma correção;
        \end{itemize}
        \item Funciona em diversas linguagens de programação além de C e C++;
    \end{itemize}
\end{frame}

\subsection{ Preparando o Executável }
\frame{\tableofcontents[
    currentsection,
    currentsubsection,
    subsectionstyle=show/shaded/hide
]}
\begin{frame}[fragile]{Como utilizá-lo?}

Geralmente compilamos assim:

\begin{center}
    \small
    \texttt{ \textbf{ gcc [{\color{blue} flags}]  $<${\color{dartmouthgreen} arquivos fonte}$>$ -o $<${\color{dartmouthgreen} saida}$>$}}
\end{center}
\\[2.0em]
Exemplo:
\begin{center}
    \small
    \texttt{ \textbf{ gcc {\color{blue} -Wall}  {\color{dartmouthgreen} main.c} -o {\color{dartmouthgreen} main} }}
\end{center}

Para usar o gdb precisamos compilar usando {\color{red} \textbf{-g}} para habilitar o suporte a debug:

\begin{center}
    \small
    \texttt{ \textbf{ gcc [{\color{blue} flags}] {\color{red} -g} $<${\color{dartmouthgreen} arquivos fonte}$>$ -o $<${\color{dartmouthgreen} saida}$>$}}
\end{center}
\\[2.0em]
Exemplo:
\begin{center}
    \small
    \texttt{ \textbf{ gcc {\color{blue} -Wall}  {\color{red} -g}  {\color{dartmouthgreen} main.c} -o {\color{dartmouthgreen} main} }}
\end{center}

\end{frame}

\subsection{ Usando o GDB }
\frame{\tableofcontents[
    currentsection,
    currentsubsection,
    subsectionstyle=show/shaded/hide
]}
\begin{frame}{Iniciando}
    \begin{itemize}
        \item Simplesmente digite ``gdb'' para entrar na interface. \texttt{\textbf{(gdb)}}
        \item É possível selecionar o executável que você deseja depurar de duas formas:
        \begin{enumerate}
            \item \textbf{ \texttt{gdb $<${\color{dartmouthgreen} arquivo compilado}$>$} }
            \item Após entrar na interface digitar: \textbf{\texttt{(gdb) {\color{blue} file} $<${\color{dartmouthgreen}arquivo compilado}$>$}}
        \end{enumerate}
    \end{itemize}
\end{frame}


\begin{frame}{Executando o programa}
    Para executar o programa basta executar:
    \begin{center}
        \small
        \texttt{ \textbf{ (gdb) {\color{dartmouthgreen} run}}}
        \\ \texttt{ \textbf{ ou } } \\
        \texttt{ \textbf{ (gdb) {\color{dartmouthgreen} r}}}
    \end{center}
    
    Se o programa não tiver nenhum erro, ele vai executar normalmente como deveria. Caso contrário ele ira retornar alguma informação que possa ser útil para localizar o erro, como:
    
    \begin{itemize}
        \item Linha que o programa estava executando quando parou;
        \item Função que estava executando;
        \item Paramêtros que a função recebeu;
        \item Entre outros;
    \end{itemize}
\end{frame}

\section{ \scshape Depurando o Programa }
\frame{\tableofcontents[
    currentsection,
    currentsubsection,
    subsectionstyle=show/shaded/hide
]}
\begin{frame}{ Depurando o Programa }
    Caso seu programa não esteja executando corretamente, com o gdb você pode investigar o problema de diversas formas, como:
    \begin{itemize}
        \item Executar o programa passo a passo;
        \item Utilizar pontos de parada;
        \item Utilizar pontos de parada condicional;
        \item Perseguindo Variáveis;
        \item Consultando e Modificando o conteúdo das variáveis em tempo de execução;
        \item Entre outros;
    \end{itemize}
\end{frame}

\subsection{ Utilizando Pontos de Parada }
\frame{\tableofcontents[
    currentsection,
    currentsubsection,
    subsectionstyle=show/shaded/hide
]}
\begin{frame}{Utilizando Pontos de Parada}

    Pontos de Parada ou \textit{Breakpoints} podem ser usados para parar o programa no meio da sua execução em um determinado ponto. Esse ponto pode ser determinado pelo comando \texttt{\textbf{{\color{blue}break}}}. O comando pode ser combinado e utilizado de diversas formas, indicando que o programa pare em uma linha, em uma função ou quando uma condição for satisfeita.
    
    Exemplo, parando em uma linha específica:
    
    \begin{center}
        \small
        \texttt{ \textbf{ (gdb) {\color{blue}break}  {\color{black}arqfonte}{\color{red}.c}:$<${\color{dartmouthgreen}line}$>$ }}
    \end{center}
\end{frame}

\begin{frame}[fragile]{Utilizando Pontos de Parada}

    Exemplo, parando em uma função específica:
    
    Suponhamos que a função abaixo esteja definida:
    
    \begin{lstlisting}
        int func(char *name, int *age);
    \end{lstlisting}
    
    Podemos parar sempre que esta função for chamada desta forma:
    \begin{center}
        \small
        \texttt{ \textbf{ (gdb) {\color{blue}break} {\color{red} func} } }
    \end{center}
\end{frame}

\begin{frame}{Depois de Parar em um Ponto de Parada}
    Depois de parar em um ponto de parada você pode fazer várias coisas, como:
    \begin{itemize}
        \item Continuar até o próximo ponto de parada;
        \begin{center}
            \small
            \texttt{ \textbf{ (gdb) {\color{blue}continue} } }
        \end{center}
        \item PExecuta a linha que está e vai para a próxima:
        \begin{center}
            \small
            \texttt{ \textbf{ (gdb) {\color{blue} next} } }
        \end{center}
        \item Executa a linha que está e vai para a próxima, se tiver chamada de função entra na função:
        \begin{center}
            \small
            \texttt{ \textbf{ (gdb) {\color{blue} step} } }
        \end{center}
        \item Avaliar o valor das variáveis;
        \begin{center}
            \small
            \texttt{ \textbf{ (gdb) {\color{blue} print} $<${\color{red} var}$>$ } }
        \end{center}
        \item Entre outros;
    \end{itemize}
    
\end{frame}

\subsection{ Pontos de Parada Condicionais }
\frame{\tableofcontents[
    currentsection,
    currentsubsection,
    subsectionstyle=show/shaded/hide
]}
\begin{frame}[fragile]{Pontos de Parada Condicionais}
    Permite definir um ponto de parada e só parar quando uma condição específica ocorrer.
    
    Exemplo:
    
    Suponhamos o seguinte trecho de código:
    
    \begin{lstlisting}
    ...
    int a = 5, b = 6;
    int n = a + b;
    int *vet = (int *) malloc ( sizeof(int) * n);
    ...
    \end{lstlisting}
    
    \begin{center}
        \small
        \texttt{ \textbf{ (gdb) {\color{blue}break}  {\color{black}arqfonte}{\color{red}.c}:{\color{dartmouthgreen}4} {\color{red} if n > 10 }}}
    \end{center}
\end{frame}

\subsection{ Consultando Variáveis e Modificando seu Conteúdo }
\frame{\tableofcontents[
    currentsection,
    currentsubsection,
    subsectionstyle=show/shaded/hide
]}
\begin{frame}[fragile]{Consultando Variáveis}

    É possível consultar o conteúdo de uma variável da seguinte forma:
    
    \begin{center}
        \small
        \texttt{ \textbf{ (gdb) {\color{blue} print} $<${\color{red} var}$>$ }}
    \end{center}
    Se a variável for um ponteiro é possível resolvê-lo antes de imprimir. De forma similar é possível acessar valor específicos de uma estrutura.

\end{frame}

\begin{frame}[fragile]{Consultando Variáveis}
    Outra forma de ver o conteúdo de uma variável, por exemplo vetor é da seguinte forma, suponhamos o trecho de alocação abaixo:
    
    \begin{lstlisting}
    ...
    int *vec = (int *) malloc ( len * sizeof(int) );
    ...
    \end{lstlisting}
    
    \begin{center}
        \small
        \texttt{ \textbf{ (gdb) {\color{blue} print} {\color{red} *vec@len} }}
    \end{center}
\end{frame}

\begin{frame}[fragile]{Consultando Variáveis}
    Agora imagine que você tem o seguinte trecho de código:
    \begin{lstlisting}
    ...
    n = 5;
    int *vec = (int *) malloc ( n * sizeof(int) );
    ...
    n = 32;
    ...
    for (int i = 0; i < n; i++ )
        printf("%d ", vec[i]);
    ...
    \end{lstlisting}
    
    Ao executar o programa você recebe o erro de \textit{segmentation fault}. É possível analisar desta forma: 
    \begin{center}
        \small
        \texttt{ \textbf{ (gdb) {\color{blue} print} {\color{red} *vec@n} }}
    \end{center}
    Caso n for maior que a capacidade do vetor alocado, possivelmente o gdb irá anunciar que a posição especificada não pode ser acessada.
\end{frame}

\begin{frame}{Alterando o Conteúdo de uma Variável}

    Para alterar o conteúdo de uma variável basta utilizar o comando:
    
    \begin{center}
        \small
        \texttt{ \textbf{ (gdb) {\color{blue} set variable} {\color{red} var = } $<${\color{dartmouthgreen}value}$>$ }}
    \end{center}

    Por exemplo:
    
    \begin{center}
        \small
        \texttt{ \textbf{ (gdb) {\color{blue} set variable} {\color{red} i = } {\color{dartmouthgreen}10} }}
    \end{center}
    
\end{frame}

\subsection{ Perseguindo Variáveis }
\frame{\tableofcontents[
    currentsection,
    currentsubsection,
    subsectionstyle=show/shaded/hide
]}
\begin{frame}{Perseguindo Variáveis}

    Perseguindo Variáveis é um título que o gdb define como \textit{Watchpoints}. \textit{Watchpoints} permitem você acompanhar uma variável durante toda a execução do programa, parando a execução \textbf{sempre} que o conteúdo desta variável for modificado.
    
    O comando é:
    
    \begin{center}
        \small
        \texttt{ \textbf{ (gdb) {\color{blue} watch} {\color{red} var} } }
    \end{center}

\end{frame}

\begin{frame}{Outras Funcionalidades Básicas}

    Roda o programa até que a função seja finalizada.
    
    \begin{center}
        \small
        \texttt{ \textbf{ (gdb) {\color{blue} finish} } }
    \end{center}

    Mostra as informações de todos \textit{Breakpoints} declarados:
     \begin{center}
        \small
        \texttt{ \textbf{ (gdb) {\color{blue} info breakpoints} } }
    \end{center}

    Para sair do gdb basta:
    
    \begin{center}
        \small
        \texttt{ \textbf{ (gdb) {\color{blue} quit} } }
        \\ ou \\
        \texttt{ \textbf{ (gdb) {\color{blue} q} } }
    \end{center}
    
    Outras informações, comandos e mais detalhes são encontrados na {\color{blue}\textbf{ \underline{\href{https://sourceware.org/gdb/current/onlinedocs/gdb/}{documentação completa do gdb}}}}.
    
\end{frame}
